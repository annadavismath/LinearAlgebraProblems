\documentclass{ximera}
%% You can put user macros here
%% However, you cannot make new environments

\listfiles

\graphicspath{
{./}
{./LTR-0070/}
{./VEC-0060/}
{./APP-0020/}
}

\usepackage{tikz}
\usepackage{tkz-euclide}
\usepackage{tikz-3dplot}
\usepackage{tikz-cd}
\usetikzlibrary{shapes.geometric}
\usetikzlibrary{arrows}
%\usetkzobj{all}
\pgfplotsset{compat=1.13} % prevents compile error.

%\renewcommand{\vec}[1]{\mathbf{#1}}
\renewcommand{\vec}{\mathbf}
\newcommand{\RR}{\mathbb{R}}
\newcommand{\dfn}{\textit}
\newcommand{\dotp}{\cdot}
\newcommand{\id}{\text{id}}
\newcommand\norm[1]{\left\lVert#1\right\rVert}
 
\newtheorem{general}{Generalization}
\newtheorem{initprob}{Exploration Problem}
\newtheorem{blankBox}{}

\tikzstyle geometryDiagrams=[ultra thick,color=blue!50!black]

%\DefineVerbatimEnvironment{octave}{Verbatim}{numbers=left,frame=lines,label=Octave,labelposition=topline}



\usepackage{mathtools}

\author{}
\license{Creative Commons 4.0 By-NC-SA}
%\outcome{Compute an antiderivative using basic formulas}
\begin{document}
\begin{exercise}

True or False?  If False, you should come up with a counterexample.  If True, can you give a proof?

 \begin{enumerate}
 \item $T : \RR^2 \to \RR^2$, given by $T(x, y) = (x, -y)$, is a linear transformation.

 \begin{multipleChoice}
 \choice[correct]{True}
 \choice{False}
 \end{multipleChoice}

 \item $T : \RR^n \to \RR$, given by $T(\vec{x}) = \vec{x} \cdot \vec{z}$ for some fixed vector $\vec{z} \in \RR^n$, is a linear transformation.

 \begin{multipleChoice}
 \choice[correct]{True}
 \choice{False}
 \end{multipleChoice}

\item $T : \RR \to \RR$, given by $T(x) = x^2$, is a linear transformation.

 \begin{multipleChoice}
 \choice{True}
 \choice[correct]{False}
 \end{multipleChoice}

 \item Let $T : \RR^n \to \RR^m$ be a linear transformation and let $\vec{v}_{1}, \dots, \vec{v}_{k}$ denote vectors in $\RR^n$.  If $\{T(\vec{v}_{1}), \dots, T(\vec{v}_{k})\}$ is linearly independent, then $\{\vec{v}_{1}, \dots, \vec{v}_{k}\}$ is also linearly independent.

 \begin{multipleChoice}
 \choice[correct]{True}
 \choice{False}
 \end{multipleChoice}

 \item Let $T : \RR^2 \to \RR^2$ be a linear transformation and suppose $\vec{v}_{1}, \vec{v}_{2}$ denote vectors in $\RR^2$.  If $\{\vec{v}_{1}, \vec{v}_{2}\}$ is linearly independent, then $\{T(\vec{v}_{1}), T(\vec{v}_{2})\}$ is also linearly independent.

 \begin{multipleChoice}
 \choice{True}
 \choice[correct]{False}
 \end{multipleChoice}
 \end{enumerate}

 
\end{exercise}

\subsection*{Source}
[Nicholson] W. Keith Nicholson, {\it Linear Algebra with Applications}, Lyryx 2021, Open Edition, Exercises 7.1.1, 7.1.2, and 7.1.17, 
\end{document}